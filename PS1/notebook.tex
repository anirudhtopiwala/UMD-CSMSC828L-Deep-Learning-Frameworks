
% Default to the notebook output style

    


% Inherit from the specified cell style.




    
\documentclass[11pt]{article}

    
    
    \usepackage[T1]{fontenc}
    % Nicer default font (+ math font) than Computer Modern for most use cases
    \usepackage{mathpazo}

    % Basic figure setup, for now with no caption control since it's done
    % automatically by Pandoc (which extracts ![](path) syntax from Markdown).
    \usepackage{graphicx}
    % We will generate all images so they have a width \maxwidth. This means
    % that they will get their normal width if they fit onto the page, but
    % are scaled down if they would overflow the margins.
    \makeatletter
    \def\maxwidth{\ifdim\Gin@nat@width>\linewidth\linewidth
    \else\Gin@nat@width\fi}
    \makeatother
    \let\Oldincludegraphics\includegraphics
    % Set max figure width to be 80% of text width, for now hardcoded.
    \renewcommand{\includegraphics}[1]{\Oldincludegraphics[width=.8\maxwidth]{#1}}
    % Ensure that by default, figures have no caption (until we provide a
    % proper Figure object with a Caption API and a way to capture that
    % in the conversion process - todo).
    \usepackage{caption}
    \DeclareCaptionLabelFormat{nolabel}{}
    \captionsetup{labelformat=nolabel}

    \usepackage{adjustbox} % Used to constrain images to a maximum size 
    \usepackage{xcolor} % Allow colors to be defined
    \usepackage{enumerate} % Needed for markdown enumerations to work
    \usepackage{geometry} % Used to adjust the document margins
    \usepackage{amsmath} % Equations
    \usepackage{amssymb} % Equations
    \usepackage{textcomp} % defines textquotesingle
    % Hack from http://tex.stackexchange.com/a/47451/13684:
    \AtBeginDocument{%
        \def\PYZsq{\textquotesingle}% Upright quotes in Pygmentized code
    }
    \usepackage{upquote} % Upright quotes for verbatim code
    \usepackage{eurosym} % defines \euro
    \usepackage[mathletters]{ucs} % Extended unicode (utf-8) support
    \usepackage[utf8x]{inputenc} % Allow utf-8 characters in the tex document
    \usepackage{fancyvrb} % verbatim replacement that allows latex
    \usepackage{grffile} % extends the file name processing of package graphics 
                         % to support a larger range 
    % The hyperref package gives us a pdf with properly built
    % internal navigation ('pdf bookmarks' for the table of contents,
    % internal cross-reference links, web links for URLs, etc.)
    \usepackage{hyperref}
    \usepackage{longtable} % longtable support required by pandoc >1.10
    \usepackage{booktabs}  % table support for pandoc > 1.12.2
    \usepackage[inline]{enumitem} % IRkernel/repr support (it uses the enumerate* environment)
    \usepackage[normalem]{ulem} % ulem is needed to support strikethroughs (\sout)
                                % normalem makes italics be italics, not underlines
    

    
    
    % Colors for the hyperref package
    \definecolor{urlcolor}{rgb}{0,.145,.698}
    \definecolor{linkcolor}{rgb}{.71,0.21,0.01}
    \definecolor{citecolor}{rgb}{.12,.54,.11}

    % ANSI colors
    \definecolor{ansi-black}{HTML}{3E424D}
    \definecolor{ansi-black-intense}{HTML}{282C36}
    \definecolor{ansi-red}{HTML}{E75C58}
    \definecolor{ansi-red-intense}{HTML}{B22B31}
    \definecolor{ansi-green}{HTML}{00A250}
    \definecolor{ansi-green-intense}{HTML}{007427}
    \definecolor{ansi-yellow}{HTML}{DDB62B}
    \definecolor{ansi-yellow-intense}{HTML}{B27D12}
    \definecolor{ansi-blue}{HTML}{208FFB}
    \definecolor{ansi-blue-intense}{HTML}{0065CA}
    \definecolor{ansi-magenta}{HTML}{D160C4}
    \definecolor{ansi-magenta-intense}{HTML}{A03196}
    \definecolor{ansi-cyan}{HTML}{60C6C8}
    \definecolor{ansi-cyan-intense}{HTML}{258F8F}
    \definecolor{ansi-white}{HTML}{C5C1B4}
    \definecolor{ansi-white-intense}{HTML}{A1A6B2}

    % commands and environments needed by pandoc snippets
    % extracted from the output of `pandoc -s`
    \providecommand{\tightlist}{%
      \setlength{\itemsep}{0pt}\setlength{\parskip}{0pt}}
    \DefineVerbatimEnvironment{Highlighting}{Verbatim}{commandchars=\\\{\}}
    % Add ',fontsize=\small' for more characters per line
    \newenvironment{Shaded}{}{}
    \newcommand{\KeywordTok}[1]{\textcolor[rgb]{0.00,0.44,0.13}{\textbf{{#1}}}}
    \newcommand{\DataTypeTok}[1]{\textcolor[rgb]{0.56,0.13,0.00}{{#1}}}
    \newcommand{\DecValTok}[1]{\textcolor[rgb]{0.25,0.63,0.44}{{#1}}}
    \newcommand{\BaseNTok}[1]{\textcolor[rgb]{0.25,0.63,0.44}{{#1}}}
    \newcommand{\FloatTok}[1]{\textcolor[rgb]{0.25,0.63,0.44}{{#1}}}
    \newcommand{\CharTok}[1]{\textcolor[rgb]{0.25,0.44,0.63}{{#1}}}
    \newcommand{\StringTok}[1]{\textcolor[rgb]{0.25,0.44,0.63}{{#1}}}
    \newcommand{\CommentTok}[1]{\textcolor[rgb]{0.38,0.63,0.69}{\textit{{#1}}}}
    \newcommand{\OtherTok}[1]{\textcolor[rgb]{0.00,0.44,0.13}{{#1}}}
    \newcommand{\AlertTok}[1]{\textcolor[rgb]{1.00,0.00,0.00}{\textbf{{#1}}}}
    \newcommand{\FunctionTok}[1]{\textcolor[rgb]{0.02,0.16,0.49}{{#1}}}
    \newcommand{\RegionMarkerTok}[1]{{#1}}
    \newcommand{\ErrorTok}[1]{\textcolor[rgb]{1.00,0.00,0.00}{\textbf{{#1}}}}
    \newcommand{\NormalTok}[1]{{#1}}
    
    % Additional commands for more recent versions of Pandoc
    \newcommand{\ConstantTok}[1]{\textcolor[rgb]{0.53,0.00,0.00}{{#1}}}
    \newcommand{\SpecialCharTok}[1]{\textcolor[rgb]{0.25,0.44,0.63}{{#1}}}
    \newcommand{\VerbatimStringTok}[1]{\textcolor[rgb]{0.25,0.44,0.63}{{#1}}}
    \newcommand{\SpecialStringTok}[1]{\textcolor[rgb]{0.73,0.40,0.53}{{#1}}}
    \newcommand{\ImportTok}[1]{{#1}}
    \newcommand{\DocumentationTok}[1]{\textcolor[rgb]{0.73,0.13,0.13}{\textit{{#1}}}}
    \newcommand{\AnnotationTok}[1]{\textcolor[rgb]{0.38,0.63,0.69}{\textbf{\textit{{#1}}}}}
    \newcommand{\CommentVarTok}[1]{\textcolor[rgb]{0.38,0.63,0.69}{\textbf{\textit{{#1}}}}}
    \newcommand{\VariableTok}[1]{\textcolor[rgb]{0.10,0.09,0.49}{{#1}}}
    \newcommand{\ControlFlowTok}[1]{\textcolor[rgb]{0.00,0.44,0.13}{\textbf{{#1}}}}
    \newcommand{\OperatorTok}[1]{\textcolor[rgb]{0.40,0.40,0.40}{{#1}}}
    \newcommand{\BuiltInTok}[1]{{#1}}
    \newcommand{\ExtensionTok}[1]{{#1}}
    \newcommand{\PreprocessorTok}[1]{\textcolor[rgb]{0.74,0.48,0.00}{{#1}}}
    \newcommand{\AttributeTok}[1]{\textcolor[rgb]{0.49,0.56,0.16}{{#1}}}
    \newcommand{\InformationTok}[1]{\textcolor[rgb]{0.38,0.63,0.69}{\textbf{\textit{{#1}}}}}
    \newcommand{\WarningTok}[1]{\textcolor[rgb]{0.38,0.63,0.69}{\textbf{\textit{{#1}}}}}
    
    
    % Define a nice break command that doesn't care if a line doesn't already
    % exist.
    \def\br{\hspace*{\fill} \\* }
    % Math Jax compatability definitions
    \def\gt{>}
    \def\lt{<}
    % Document parameters
    \title{Q1}
    
    
    

    % Pygments definitions
    
\makeatletter
\def\PY@reset{\let\PY@it=\relax \let\PY@bf=\relax%
    \let\PY@ul=\relax \let\PY@tc=\relax%
    \let\PY@bc=\relax \let\PY@ff=\relax}
\def\PY@tok#1{\csname PY@tok@#1\endcsname}
\def\PY@toks#1+{\ifx\relax#1\empty\else%
    \PY@tok{#1}\expandafter\PY@toks\fi}
\def\PY@do#1{\PY@bc{\PY@tc{\PY@ul{%
    \PY@it{\PY@bf{\PY@ff{#1}}}}}}}
\def\PY#1#2{\PY@reset\PY@toks#1+\relax+\PY@do{#2}}

\expandafter\def\csname PY@tok@w\endcsname{\def\PY@tc##1{\textcolor[rgb]{0.73,0.73,0.73}{##1}}}
\expandafter\def\csname PY@tok@c\endcsname{\let\PY@it=\textit\def\PY@tc##1{\textcolor[rgb]{0.25,0.50,0.50}{##1}}}
\expandafter\def\csname PY@tok@cp\endcsname{\def\PY@tc##1{\textcolor[rgb]{0.74,0.48,0.00}{##1}}}
\expandafter\def\csname PY@tok@k\endcsname{\let\PY@bf=\textbf\def\PY@tc##1{\textcolor[rgb]{0.00,0.50,0.00}{##1}}}
\expandafter\def\csname PY@tok@kp\endcsname{\def\PY@tc##1{\textcolor[rgb]{0.00,0.50,0.00}{##1}}}
\expandafter\def\csname PY@tok@kt\endcsname{\def\PY@tc##1{\textcolor[rgb]{0.69,0.00,0.25}{##1}}}
\expandafter\def\csname PY@tok@o\endcsname{\def\PY@tc##1{\textcolor[rgb]{0.40,0.40,0.40}{##1}}}
\expandafter\def\csname PY@tok@ow\endcsname{\let\PY@bf=\textbf\def\PY@tc##1{\textcolor[rgb]{0.67,0.13,1.00}{##1}}}
\expandafter\def\csname PY@tok@nb\endcsname{\def\PY@tc##1{\textcolor[rgb]{0.00,0.50,0.00}{##1}}}
\expandafter\def\csname PY@tok@nf\endcsname{\def\PY@tc##1{\textcolor[rgb]{0.00,0.00,1.00}{##1}}}
\expandafter\def\csname PY@tok@nc\endcsname{\let\PY@bf=\textbf\def\PY@tc##1{\textcolor[rgb]{0.00,0.00,1.00}{##1}}}
\expandafter\def\csname PY@tok@nn\endcsname{\let\PY@bf=\textbf\def\PY@tc##1{\textcolor[rgb]{0.00,0.00,1.00}{##1}}}
\expandafter\def\csname PY@tok@ne\endcsname{\let\PY@bf=\textbf\def\PY@tc##1{\textcolor[rgb]{0.82,0.25,0.23}{##1}}}
\expandafter\def\csname PY@tok@nv\endcsname{\def\PY@tc##1{\textcolor[rgb]{0.10,0.09,0.49}{##1}}}
\expandafter\def\csname PY@tok@no\endcsname{\def\PY@tc##1{\textcolor[rgb]{0.53,0.00,0.00}{##1}}}
\expandafter\def\csname PY@tok@nl\endcsname{\def\PY@tc##1{\textcolor[rgb]{0.63,0.63,0.00}{##1}}}
\expandafter\def\csname PY@tok@ni\endcsname{\let\PY@bf=\textbf\def\PY@tc##1{\textcolor[rgb]{0.60,0.60,0.60}{##1}}}
\expandafter\def\csname PY@tok@na\endcsname{\def\PY@tc##1{\textcolor[rgb]{0.49,0.56,0.16}{##1}}}
\expandafter\def\csname PY@tok@nt\endcsname{\let\PY@bf=\textbf\def\PY@tc##1{\textcolor[rgb]{0.00,0.50,0.00}{##1}}}
\expandafter\def\csname PY@tok@nd\endcsname{\def\PY@tc##1{\textcolor[rgb]{0.67,0.13,1.00}{##1}}}
\expandafter\def\csname PY@tok@s\endcsname{\def\PY@tc##1{\textcolor[rgb]{0.73,0.13,0.13}{##1}}}
\expandafter\def\csname PY@tok@sd\endcsname{\let\PY@it=\textit\def\PY@tc##1{\textcolor[rgb]{0.73,0.13,0.13}{##1}}}
\expandafter\def\csname PY@tok@si\endcsname{\let\PY@bf=\textbf\def\PY@tc##1{\textcolor[rgb]{0.73,0.40,0.53}{##1}}}
\expandafter\def\csname PY@tok@se\endcsname{\let\PY@bf=\textbf\def\PY@tc##1{\textcolor[rgb]{0.73,0.40,0.13}{##1}}}
\expandafter\def\csname PY@tok@sr\endcsname{\def\PY@tc##1{\textcolor[rgb]{0.73,0.40,0.53}{##1}}}
\expandafter\def\csname PY@tok@ss\endcsname{\def\PY@tc##1{\textcolor[rgb]{0.10,0.09,0.49}{##1}}}
\expandafter\def\csname PY@tok@sx\endcsname{\def\PY@tc##1{\textcolor[rgb]{0.00,0.50,0.00}{##1}}}
\expandafter\def\csname PY@tok@m\endcsname{\def\PY@tc##1{\textcolor[rgb]{0.40,0.40,0.40}{##1}}}
\expandafter\def\csname PY@tok@gh\endcsname{\let\PY@bf=\textbf\def\PY@tc##1{\textcolor[rgb]{0.00,0.00,0.50}{##1}}}
\expandafter\def\csname PY@tok@gu\endcsname{\let\PY@bf=\textbf\def\PY@tc##1{\textcolor[rgb]{0.50,0.00,0.50}{##1}}}
\expandafter\def\csname PY@tok@gd\endcsname{\def\PY@tc##1{\textcolor[rgb]{0.63,0.00,0.00}{##1}}}
\expandafter\def\csname PY@tok@gi\endcsname{\def\PY@tc##1{\textcolor[rgb]{0.00,0.63,0.00}{##1}}}
\expandafter\def\csname PY@tok@gr\endcsname{\def\PY@tc##1{\textcolor[rgb]{1.00,0.00,0.00}{##1}}}
\expandafter\def\csname PY@tok@ge\endcsname{\let\PY@it=\textit}
\expandafter\def\csname PY@tok@gs\endcsname{\let\PY@bf=\textbf}
\expandafter\def\csname PY@tok@gp\endcsname{\let\PY@bf=\textbf\def\PY@tc##1{\textcolor[rgb]{0.00,0.00,0.50}{##1}}}
\expandafter\def\csname PY@tok@go\endcsname{\def\PY@tc##1{\textcolor[rgb]{0.53,0.53,0.53}{##1}}}
\expandafter\def\csname PY@tok@gt\endcsname{\def\PY@tc##1{\textcolor[rgb]{0.00,0.27,0.87}{##1}}}
\expandafter\def\csname PY@tok@err\endcsname{\def\PY@bc##1{\setlength{\fboxsep}{0pt}\fcolorbox[rgb]{1.00,0.00,0.00}{1,1,1}{\strut ##1}}}
\expandafter\def\csname PY@tok@kc\endcsname{\let\PY@bf=\textbf\def\PY@tc##1{\textcolor[rgb]{0.00,0.50,0.00}{##1}}}
\expandafter\def\csname PY@tok@kd\endcsname{\let\PY@bf=\textbf\def\PY@tc##1{\textcolor[rgb]{0.00,0.50,0.00}{##1}}}
\expandafter\def\csname PY@tok@kn\endcsname{\let\PY@bf=\textbf\def\PY@tc##1{\textcolor[rgb]{0.00,0.50,0.00}{##1}}}
\expandafter\def\csname PY@tok@kr\endcsname{\let\PY@bf=\textbf\def\PY@tc##1{\textcolor[rgb]{0.00,0.50,0.00}{##1}}}
\expandafter\def\csname PY@tok@bp\endcsname{\def\PY@tc##1{\textcolor[rgb]{0.00,0.50,0.00}{##1}}}
\expandafter\def\csname PY@tok@fm\endcsname{\def\PY@tc##1{\textcolor[rgb]{0.00,0.00,1.00}{##1}}}
\expandafter\def\csname PY@tok@vc\endcsname{\def\PY@tc##1{\textcolor[rgb]{0.10,0.09,0.49}{##1}}}
\expandafter\def\csname PY@tok@vg\endcsname{\def\PY@tc##1{\textcolor[rgb]{0.10,0.09,0.49}{##1}}}
\expandafter\def\csname PY@tok@vi\endcsname{\def\PY@tc##1{\textcolor[rgb]{0.10,0.09,0.49}{##1}}}
\expandafter\def\csname PY@tok@vm\endcsname{\def\PY@tc##1{\textcolor[rgb]{0.10,0.09,0.49}{##1}}}
\expandafter\def\csname PY@tok@sa\endcsname{\def\PY@tc##1{\textcolor[rgb]{0.73,0.13,0.13}{##1}}}
\expandafter\def\csname PY@tok@sb\endcsname{\def\PY@tc##1{\textcolor[rgb]{0.73,0.13,0.13}{##1}}}
\expandafter\def\csname PY@tok@sc\endcsname{\def\PY@tc##1{\textcolor[rgb]{0.73,0.13,0.13}{##1}}}
\expandafter\def\csname PY@tok@dl\endcsname{\def\PY@tc##1{\textcolor[rgb]{0.73,0.13,0.13}{##1}}}
\expandafter\def\csname PY@tok@s2\endcsname{\def\PY@tc##1{\textcolor[rgb]{0.73,0.13,0.13}{##1}}}
\expandafter\def\csname PY@tok@sh\endcsname{\def\PY@tc##1{\textcolor[rgb]{0.73,0.13,0.13}{##1}}}
\expandafter\def\csname PY@tok@s1\endcsname{\def\PY@tc##1{\textcolor[rgb]{0.73,0.13,0.13}{##1}}}
\expandafter\def\csname PY@tok@mb\endcsname{\def\PY@tc##1{\textcolor[rgb]{0.40,0.40,0.40}{##1}}}
\expandafter\def\csname PY@tok@mf\endcsname{\def\PY@tc##1{\textcolor[rgb]{0.40,0.40,0.40}{##1}}}
\expandafter\def\csname PY@tok@mh\endcsname{\def\PY@tc##1{\textcolor[rgb]{0.40,0.40,0.40}{##1}}}
\expandafter\def\csname PY@tok@mi\endcsname{\def\PY@tc##1{\textcolor[rgb]{0.40,0.40,0.40}{##1}}}
\expandafter\def\csname PY@tok@il\endcsname{\def\PY@tc##1{\textcolor[rgb]{0.40,0.40,0.40}{##1}}}
\expandafter\def\csname PY@tok@mo\endcsname{\def\PY@tc##1{\textcolor[rgb]{0.40,0.40,0.40}{##1}}}
\expandafter\def\csname PY@tok@ch\endcsname{\let\PY@it=\textit\def\PY@tc##1{\textcolor[rgb]{0.25,0.50,0.50}{##1}}}
\expandafter\def\csname PY@tok@cm\endcsname{\let\PY@it=\textit\def\PY@tc##1{\textcolor[rgb]{0.25,0.50,0.50}{##1}}}
\expandafter\def\csname PY@tok@cpf\endcsname{\let\PY@it=\textit\def\PY@tc##1{\textcolor[rgb]{0.25,0.50,0.50}{##1}}}
\expandafter\def\csname PY@tok@c1\endcsname{\let\PY@it=\textit\def\PY@tc##1{\textcolor[rgb]{0.25,0.50,0.50}{##1}}}
\expandafter\def\csname PY@tok@cs\endcsname{\let\PY@it=\textit\def\PY@tc##1{\textcolor[rgb]{0.25,0.50,0.50}{##1}}}

\def\PYZbs{\char`\\}
\def\PYZus{\char`\_}
\def\PYZob{\char`\{}
\def\PYZcb{\char`\}}
\def\PYZca{\char`\^}
\def\PYZam{\char`\&}
\def\PYZlt{\char`\<}
\def\PYZgt{\char`\>}
\def\PYZsh{\char`\#}
\def\PYZpc{\char`\%}
\def\PYZdl{\char`\$}
\def\PYZhy{\char`\-}
\def\PYZsq{\char`\'}
\def\PYZdq{\char`\"}
\def\PYZti{\char`\~}
% for compatibility with earlier versions
\def\PYZat{@}
\def\PYZlb{[}
\def\PYZrb{]}
\makeatother


    % Exact colors from NB
    \definecolor{incolor}{rgb}{0.0, 0.0, 0.5}
    \definecolor{outcolor}{rgb}{0.545, 0.0, 0.0}



    
    % Prevent overflowing lines due to hard-to-break entities
    \sloppy 
    % Setup hyperref package
    \hypersetup{
      breaklinks=true,  % so long urls are correctly broken across lines
      colorlinks=true,
      urlcolor=urlcolor,
      linkcolor=linkcolor,
      citecolor=citecolor,
      }
    % Slightly bigger margins than the latex defaults
    
    \geometry{verbose,tmargin=1in,bmargin=1in,lmargin=1in,rmargin=1in}
    
    

    \begin{document}
    
    
    \maketitle
    
    

    
    \subsection{Question 1: Implementation of basic neural netwrok without
any hidden
layer.}\label{question-1-implementation-of-basic-neural-netwrok-without-any-hidden-layer.}

\begin{enumerate}
\def\labelenumi{(\alph{enumi})}
\tightlist
\item
  Generate some random 1D test data according to a simple linear
  function, with Gaussian noise added. For example, your data might be
  generated with: y = 7x + 3 + ξ, where ξ is a Gaussian random variable.
  Include a plot showing the training data and the function that your
  network computes. (You can plot the function by evaluating it on a
  range of different inputs). This is all 1D, so easy to visualize.
\end{enumerate}

    \textbf{Sol:} Here we implement designing a neural network which will
take arbitary number of inputs for arbitary dimensions. This network has
no hidden layers. \#\#\#\# Loss Function Regression loss function as
given in the question.

\paragraph{Input}\label{input}

Inputs are generated randomly by setting the number of training inputs
and the dimensions.

\paragraph{Outputs}\label{outputs}

The oputputs are limited to a single output. If the there is an n
dimensional input, output is a function of these n inputs to produce a
single value. (Here I have used summation to add the input for n
dimensions to generate a single output). I have used the function y = 7x
+ 3 + ξ to produce all the outputs.

\paragraph{Hyper Paramters}\label{hyper-paramters}

For this function there are only two Hyper paramters, that is the
\textbf{Learning Rate} (Step Size) and the number of \textbf{epochs}

\textbf{Implicit Hyper paramters:} weights and bias are initialized to a
random value.

\paragraph{The Netwrok has 5 main
Functions}\label{the-netwrok-has-5-main-functions}

\begin{enumerate}
\def\labelenumi{\arabic{enumi})}
\tightlist
\item
  Network Initialization
\item
  feed\_forward
\item
  back\_prop
\item
  update\_network
\item
  train
\end{enumerate}

\textbf{Now lets got thorugh each function}

    \begin{Verbatim}[commandchars=\\\{\}]
{\color{incolor}In [{\color{incolor}2}]:} \PY{c+c1}{\PYZsh{}\PYZsh{} Initialize the Network}
        \PY{k}{def} \PY{n+nf}{initialize\PYZus{}network}\PY{p}{(}\PY{n}{n\PYZus{}inputs}\PY{p}{,} \PY{n}{dimensions}\PY{p}{)}\PY{p}{:}
            \PY{n}{x}\PY{o}{=} \PY{n}{np}\PY{o}{.}\PY{n}{random}\PY{o}{.}\PY{n}{rand}\PY{p}{(}\PY{n}{n\PYZus{}inputs}\PY{p}{,}\PY{n}{dimensions}\PY{p}{)} \PY{c+c1}{\PYZsh{}\PYZsh{} Randomly Genrated Inputs}
            \PY{n}{y}\PY{o}{=} \PY{l+m+mi}{7}\PY{o}{*}\PY{n}{x} \PY{o}{+} \PY{l+m+mi}{3} \PY{o}{+} \PY{n}{np}\PY{o}{.}\PY{n}{random}\PY{o}{.}\PY{n}{randn}\PY{p}{(}\PY{l+m+mi}{1}\PY{p}{)}        \PY{c+c1}{\PYZsh{}\PYZsh{} Output function of x}
            \PY{n}{y} \PY{o}{=} \PY{n}{np}\PY{o}{.}\PY{n}{sum}\PY{p}{(}\PY{n}{y}\PY{p}{,}\PY{n}{axis}\PY{o}{=}\PY{l+m+mi}{1}\PY{p}{,}\PY{n}{keepdims}\PY{o}{=} \PY{k+kc}{True}\PY{p}{)}    \PY{c+c1}{\PYZsh{}\PYZsh{} To produce a single output}
            \PY{n}{wts}\PY{o}{=} \PY{n}{np}\PY{o}{.}\PY{n}{random}\PY{o}{.}\PY{n}{randn}\PY{p}{(}\PY{n}{dimensions}\PY{p}{,}\PY{l+m+mi}{1}\PY{p}{)}     \PY{c+c1}{\PYZsh{}\PYZsh{} Weight Initialization}
            \PY{n}{b} \PY{o}{=}  \PY{n}{np}\PY{o}{.}\PY{n}{random}\PY{o}{.}\PY{n}{randn}\PY{p}{(}\PY{l+m+mi}{1}\PY{p}{)}                \PY{c+c1}{\PYZsh{}\PYZsh{} Bias Initialization}
            \PY{k}{return} \PY{n}{x}\PY{p}{,}\PY{n}{y}\PY{p}{,}\PY{n}{wts}\PY{p}{,}\PY{n}{b}
        
        \PY{c+c1}{\PYZsh{}\PYZsh{} Feed Forward Function}
        \PY{k}{def} \PY{n+nf}{feed\PYZus{}forward}\PY{p}{(}\PY{n}{x}\PY{p}{,}\PY{n}{wts}\PY{p}{,}\PY{n}{b}\PY{p}{)}\PY{p}{:}
            \PY{n}{a1}\PY{o}{=} \PY{n}{np}\PY{o}{.}\PY{n}{dot}\PY{p}{(}\PY{n}{x}\PY{p}{,}\PY{n}{wts}\PY{p}{)}\PY{o}{+}\PY{n}{b}                     \PY{c+c1}{\PYZsh{}\PYZsh{} calculating the Output}
            \PY{k}{return} \PY{n}{a1}
        
        \PY{c+c1}{\PYZsh{}\PYZsh{} Back Propogation}
        \PY{k}{def} \PY{n+nf}{back\PYZus{}prop}\PY{p}{(}\PY{n}{x}\PY{p}{,}\PY{n}{y}\PY{p}{,}\PY{n}{a1}\PY{p}{)}\PY{p}{:}
            \PY{n}{error} \PY{o}{=} \PY{p}{(}\PY{p}{(}\PY{p}{(}\PY{n}{y}\PY{o}{\PYZhy{}}\PY{n}{a1}\PY{p}{)}\PY{o}{*}\PY{o}{*}\PY{l+m+mi}{2}\PY{p}{)}\PY{o}{.}\PY{n}{sum}\PY{p}{(}\PY{p}{)}\PY{p}{)}             \PY{c+c1}{\PYZsh{}\PYZsh{} Calculating the error}
            \PY{n}{wt\PYZus{}err} \PY{o}{=} \PY{o}{\PYZhy{}}\PY{n}{x}\PY{o}{.}\PY{n}{T}\PY{o}{.}\PY{n}{dot}\PY{p}{(}\PY{n}{y}\PY{o}{\PYZhy{}}\PY{n}{a1}\PY{p}{)}                 \PY{c+c1}{\PYZsh{}\PYZsh{} Calculating Weight error by differentiating the cost function wrt w}
            \PY{n}{b\PYZus{}err} \PY{o}{=} \PY{p}{(}\PY{n}{a1}\PY{o}{\PYZhy{}}\PY{n}{y}\PY{p}{)}\PY{o}{.}\PY{n}{sum}\PY{p}{(}\PY{p}{)}                    \PY{c+c1}{\PYZsh{}\PYZsh{} Claculating Bias Error by differentiating the cost function wrt b}
            \PY{k}{return} \PY{n}{wt\PYZus{}err}\PY{p}{,} \PY{n}{b\PYZus{}err}\PY{p}{,} \PY{n}{error}
        
        \PY{c+c1}{\PYZsh{}\PYZsh{} Updating the Network}
        \PY{k}{def} \PY{n+nf}{update\PYZus{}network}\PY{p}{(}\PY{n}{wts}\PY{p}{,} \PY{n}{b}\PY{p}{,} \PY{n}{wt\PYZus{}err}\PY{p}{,} \PY{n}{b\PYZus{}err}\PY{p}{,}\PY{n}{lr}\PY{p}{)}\PY{p}{:}
            \PY{n}{wts} \PY{o}{\PYZhy{}}\PY{o}{=} \PY{n}{lr}\PY{o}{*}\PY{n}{wt\PYZus{}err}                        \PY{c+c1}{\PYZsh{}\PYZsh{} Updating weights }
            \PY{n}{b} \PY{o}{\PYZhy{}}\PY{o}{=} \PY{n}{lr}\PY{o}{*}\PY{n}{b\PYZus{}err}                           \PY{c+c1}{\PYZsh{}\PYZsh{} Updating b}
            \PY{k}{return} \PY{n}{wts}\PY{p}{,}\PY{n}{b}
        
        \PY{c+c1}{\PYZsh{}\PYZsh{} Training the Network}
        \PY{k}{def} \PY{n+nf}{train}\PY{p}{(}\PY{n}{net}\PY{p}{,}\PY{n}{epochs}\PY{p}{,}\PY{n}{lr}\PY{p}{)}\PY{p}{:}
            \PY{n}{error} \PY{o}{=} \PY{p}{[}\PY{p}{]}
            \PY{n}{wt} \PY{o}{=} \PY{n}{net}\PY{p}{[}\PY{l+m+mi}{2}\PY{p}{]}
            \PY{n}{b\PYZus{}}\PY{o}{=} \PY{n}{net}\PY{p}{[}\PY{l+m+mi}{3}\PY{p}{]}
            \PY{k}{for} \PY{n}{i} \PY{o+ow}{in} \PY{n+nb}{range}\PY{p}{(}\PY{n}{epochs}\PY{p}{)}\PY{p}{:}
                \PY{n}{a1} \PY{o}{=} \PY{n}{feed\PYZus{}forward}\PY{p}{(}\PY{n}{net}\PY{p}{[}\PY{l+m+mi}{0}\PY{p}{]}\PY{p}{,}\PY{n}{wt}\PY{p}{,}\PY{n}{b\PYZus{}}\PY{p}{)}
                \PY{n}{err}\PY{o}{=} \PY{n}{back\PYZus{}prop}\PY{p}{(}\PY{n}{net}\PY{p}{[}\PY{l+m+mi}{0}\PY{p}{]}\PY{p}{,} \PY{n}{net}\PY{p}{[}\PY{l+m+mi}{1}\PY{p}{]}\PY{p}{,}\PY{n}{a1}\PY{p}{)}
                \PY{n}{error}\PY{o}{.}\PY{n}{append}\PY{p}{(}\PY{n}{err}\PY{p}{[}\PY{l+m+mi}{2}\PY{p}{]}\PY{p}{)}
                \PY{n}{update} \PY{o}{=} \PY{n}{update\PYZus{}network}\PY{p}{(}\PY{n}{net}\PY{p}{[}\PY{l+m+mi}{2}\PY{p}{]}\PY{p}{,} \PY{n}{net}\PY{p}{[}\PY{l+m+mi}{3}\PY{p}{]}\PY{p}{,} \PY{n}{err}\PY{p}{[}\PY{l+m+mi}{0}\PY{p}{]}\PY{p}{,} \PY{n}{err}\PY{p}{[}\PY{l+m+mi}{1}\PY{p}{]}\PY{p}{,}\PY{n}{lr}\PY{p}{)}
                \PY{n}{wt} \PY{o}{=} \PY{n}{update}\PY{p}{[}\PY{l+m+mi}{0}\PY{p}{]}                      \PY{c+c1}{\PYZsh{}\PYZsh{} Weight Update}
                \PY{n}{b\PYZus{}} \PY{o}{=} \PY{n}{update} \PY{p}{[}\PY{l+m+mi}{1}\PY{p}{]}                     \PY{c+c1}{\PYZsh{}\PYZsh{} Bias Update}
            \PY{k}{return} \PY{n}{error}\PY{p}{,} \PY{n}{update}\PY{p}{[}\PY{l+m+mi}{0}\PY{p}{]}\PY{p}{,} \PY{n}{update}\PY{p}{[}\PY{l+m+mi}{1}\PY{p}{]}
\end{Verbatim}


    \subsection{(a) Defining the Main Function and Implementing for 1D
input}\label{a-defining-the-main-function-and-implementing-for-1d-input}

    \begin{Verbatim}[commandchars=\\\{\}]
{\color{incolor}In [{\color{incolor}62}]:} \PY{c+c1}{\PYZsh{} Importing the Libraries}
         \PY{k+kn}{import} \PY{n+nn}{numpy} \PY{k}{as} \PY{n+nn}{np}
         \PY{k+kn}{from} \PY{n+nn}{matplotlib} \PY{k}{import} \PY{n}{pyplot} \PY{k}{as} \PY{n}{plt}
         \PY{k+kn}{from} \PY{n+nn}{math} \PY{k}{import} \PY{n}{exp}
         
         \PY{c+c1}{\PYZsh{} Initializing the Network}
         \PY{n}{n\PYZus{}inputs} \PY{o}{=} \PY{l+m+mi}{1000}
         \PY{n}{net}\PY{o}{=}\PY{n}{initialize\PYZus{}network}\PY{p}{(}\PY{n}{n\PYZus{}inputs}\PY{p}{,}\PY{l+m+mi}{1}\PY{p}{)}
         
         \PY{c+c1}{\PYZsh{}\PYZsh{} Hyper Parameter Initialization}
         \PY{n}{epochs}\PY{o}{=} \PY{l+m+mi}{100}     \PY{c+c1}{\PYZsh{}\PYZsh{} Step Size}
         \PY{n}{lr} \PY{o}{=} \PY{l+m+mf}{0.001}
         \PY{n}{np}\PY{o}{.}\PY{n}{random}\PY{o}{.}\PY{n}{seed}\PY{p}{(}\PY{l+m+mi}{1}\PY{p}{)}
         
         \PY{c+c1}{\PYZsh{}\PYZsh{} Train}
         \PY{n}{tr} \PY{o}{=} \PY{n}{train} \PY{p}{(}\PY{n}{net}\PY{p}{,}\PY{n}{epochs}\PY{p}{,}\PY{n}{lr}\PY{p}{)}
\end{Verbatim}


    \subsubsection{Plot For Training Data}\label{plot-for-training-data}

\textbf{Red: Ground Truth Values }

\textbf{Blue: Predicted Values}

    \begin{Verbatim}[commandchars=\\\{\}]
{\color{incolor}In [{\color{incolor}63}]:} \PY{c+c1}{\PYZsh{}\PYZsh{} Plotting the Outputs vs Inputs}
         \PY{n}{plt}\PY{o}{.}\PY{n}{plot}\PY{p}{(}\PY{n}{net}\PY{p}{[}\PY{l+m+mi}{0}\PY{p}{]}\PY{p}{,}\PY{n}{net}\PY{p}{[}\PY{l+m+mi}{1}\PY{p}{]}\PY{p}{,}\PY{l+s+s1}{\PYZsq{}}\PY{l+s+s1}{r}\PY{l+s+s1}{\PYZsq{}}\PY{p}{)}
         
         
         \PY{c+c1}{\PYZsh{}\PYZsh{} Caluclating Predictions from the network}
         \PY{n}{plt}\PY{o}{.}\PY{n}{plot}\PY{p}{(}\PY{n}{net}\PY{p}{[}\PY{l+m+mi}{0}\PY{p}{]}\PY{p}{,} \PY{n}{feed\PYZus{}forward}\PY{p}{(}\PY{n}{net}\PY{p}{[}\PY{l+m+mi}{0}\PY{p}{]}\PY{p}{,}\PY{n}{tr}\PY{p}{[}\PY{l+m+mi}{1}\PY{p}{]}\PY{p}{,}\PY{n}{tr}\PY{p}{[}\PY{l+m+mi}{2}\PY{p}{]}\PY{p}{)}\PY{p}{,}\PY{l+s+s1}{\PYZsq{}}\PY{l+s+s1}{b}\PY{l+s+s1}{\PYZsq{}}\PY{p}{)}
         
         \PY{n}{plt}\PY{o}{.}\PY{n}{show}\PY{p}{(}\PY{p}{)}
         
         
         \PY{c+c1}{\PYZsh{} Plotting the Error}
         \PY{n}{plt}\PY{o}{.}\PY{n}{plot}\PY{p}{(}\PY{n}{tr}\PY{p}{[}\PY{l+m+mi}{0}\PY{p}{]}\PY{p}{)}
         \PY{n}{plt}\PY{o}{.}\PY{n}{show}\PY{p}{(}\PY{p}{)}
\end{Verbatim}


    \begin{center}
    \adjustimage{max size={0.9\linewidth}{0.9\paperheight}}{output_6_0.png}
    \end{center}
    { \hspace*{\fill} \\}
    
    \begin{center}
    \adjustimage{max size={0.9\linewidth}{0.9\paperheight}}{output_6_1.png}
    \end{center}
    { \hspace*{\fill} \\}
    
    \subsubsection{Generating Test Data}\label{generating-test-data}

    \begin{Verbatim}[commandchars=\\\{\}]
{\color{incolor}In [{\color{incolor}27}]:} \PY{n}{a}\PY{o}{=}\PY{n}{np}\PY{o}{.}\PY{n}{random}\PY{o}{.}\PY{n}{randint}\PY{p}{(}\PY{n}{low}\PY{o}{=}\PY{l+m+mi}{1}\PY{p}{,} \PY{n}{high}\PY{o}{=}\PY{l+m+mi}{1000}\PY{p}{,} \PY{n}{size}\PY{o}{=}\PY{l+m+mi}{2}\PY{p}{)}
         \PY{n}{x} \PY{o}{=} \PY{n}{np}\PY{o}{.}\PY{n}{reshape}\PY{p}{(}\PY{n}{np}\PY{o}{.}\PY{n}{linspace}\PY{p}{(}\PY{n}{a}\PY{p}{[}\PY{l+m+mi}{0}\PY{p}{]}\PY{p}{,}\PY{n}{a}\PY{p}{[}\PY{l+m+mi}{1}\PY{p}{]}\PY{p}{,} \PY{n}{n\PYZus{}inputs}\PY{p}{)}\PY{p}{,}\PY{p}{(}\PY{n}{n\PYZus{}inputs}\PY{p}{,}\PY{l+m+mi}{1}\PY{p}{)}\PY{p}{)} 
         \PY{n}{y} \PY{o}{=} \PY{l+m+mi}{7}\PY{o}{*}\PY{n}{x} \PY{o}{+} \PY{l+m+mi}{3} \PY{o}{+} \PY{n}{np}\PY{o}{.}\PY{n}{random}\PY{o}{.}\PY{n}{randn}\PY{p}{(}\PY{l+m+mi}{1}\PY{p}{)}
\end{Verbatim}


    \subsubsection{Plot For Test Data}\label{plot-for-test-data}

\textbf{Red: Ground Truth Values }

\textbf{Blue: Predicted Values}

    \begin{Verbatim}[commandchars=\\\{\}]
{\color{incolor}In [{\color{incolor}64}]:} \PY{c+c1}{\PYZsh{}\PYZsh{} Plotting the Outputs vs Inputs}
         \PY{n}{plt}\PY{o}{.}\PY{n}{plot}\PY{p}{(}\PY{n}{x}\PY{p}{,}\PY{n}{y}\PY{p}{,}\PY{l+s+s1}{\PYZsq{}}\PY{l+s+s1}{r}\PY{l+s+s1}{\PYZsq{}}\PY{p}{)}
         
         \PY{c+c1}{\PYZsh{}\PYZsh{} Caluclating Predictions from the network}
         \PY{n}{plt}\PY{o}{.}\PY{n}{plot}\PY{p}{(}\PY{n}{x}\PY{p}{,}\PY{n}{feed\PYZus{}forward}\PY{p}{(}\PY{n}{x}\PY{p}{,}\PY{n}{tr}\PY{p}{[}\PY{l+m+mi}{1}\PY{p}{]}\PY{p}{,}\PY{n}{tr}\PY{p}{[}\PY{l+m+mi}{2}\PY{p}{]}\PY{p}{)}\PY{p}{,}\PY{l+s+s1}{\PYZsq{}}\PY{l+s+s1}{b}\PY{l+s+s1}{\PYZsq{}}\PY{p}{)}
         
         \PY{n}{plt}\PY{o}{.}\PY{n}{show}\PY{p}{(}\PY{p}{)}
         
         
         \PY{c+c1}{\PYZsh{} Plotting the Error}
         \PY{n}{plt}\PY{o}{.}\PY{n}{plot}\PY{p}{(}\PY{n}{tr}\PY{p}{[}\PY{l+m+mi}{0}\PY{p}{]}\PY{p}{)}
         \PY{n}{plt}\PY{o}{.}\PY{n}{show}\PY{p}{(}\PY{p}{)}
\end{Verbatim}


    \begin{center}
    \adjustimage{max size={0.9\linewidth}{0.9\paperheight}}{output_10_0.png}
    \end{center}
    { \hspace*{\fill} \\}
    
    \begin{center}
    \adjustimage{max size={0.9\linewidth}{0.9\paperheight}}{output_10_1.png}
    \end{center}
    { \hspace*{\fill} \\}
    
    \subsubsection{Important Inferences}\label{important-inferences}

\begin{enumerate}
\def\labelenumi{\arabic{enumi}.}
\tightlist
\item
  For n\_inputs or number of training sample = 100 ,the training data
  start to converge from 600 epochs onwards, and totaly converge by 700
  epochs.
\item
  A learning rate or step size of 0.001 is found to be ideal
\item
  Another hyper paramter found is the number of inputs.If we change the
  inputs from 100 to 1000, the training converged in just 100 epochs
  with the same learning rate. Therefore more training data helped the
  function converge faster.
\end{enumerate}

    \subsection{(b) Defining the Main Function and Implementing for nD
input}\label{b-defining-the-main-function-and-implementing-for-nd-input}

    \begin{Verbatim}[commandchars=\\\{\}]
{\color{incolor}In [{\color{incolor}73}]:} \PY{c+c1}{\PYZsh{} Importing the Libraries}
         \PY{k+kn}{import} \PY{n+nn}{numpy} \PY{k}{as} \PY{n+nn}{np}
         \PY{k+kn}{from} \PY{n+nn}{matplotlib} \PY{k}{import} \PY{n}{pyplot} \PY{k}{as} \PY{n}{plt}
         \PY{k+kn}{from} \PY{n+nn}{math} \PY{k}{import} \PY{n}{exp}
         
         \PY{n}{np}\PY{o}{.}\PY{n}{random}\PY{o}{.}\PY{n}{seed}\PY{p}{(}\PY{l+m+mi}{1}\PY{p}{)}
         \PY{c+c1}{\PYZsh{} Initializing the Network}
         \PY{n}{n\PYZus{}inputs} \PY{o}{=} \PY{l+m+mi}{100}
         \PY{n}{dimensions}\PY{o}{=} \PY{l+m+mi}{5}
         \PY{n}{net}\PY{o}{=}\PY{n}{initialize\PYZus{}network}\PY{p}{(}\PY{n}{n\PYZus{}inputs}\PY{p}{,}\PY{n}{dimensions}\PY{p}{)}
         
         \PY{c+c1}{\PYZsh{}\PYZsh{} Hyper Parameter Initialization}
         \PY{n}{epochs}\PY{o}{=} \PY{l+m+mi}{1000}    \PY{c+c1}{\PYZsh{}\PYZsh{} Step Size}
         \PY{n}{lr} \PY{o}{=} \PY{l+m+mf}{0.001}
         \PY{n}{np}\PY{o}{.}\PY{n}{random}\PY{o}{.}\PY{n}{seed}\PY{p}{(}\PY{l+m+mi}{1}\PY{p}{)}
         
         \PY{c+c1}{\PYZsh{}\PYZsh{} Train}
         \PY{n}{tr} \PY{o}{=} \PY{n}{train} \PY{p}{(}\PY{n}{net}\PY{p}{,}\PY{n}{epochs}\PY{p}{,}\PY{n}{lr}\PY{p}{)}
\end{Verbatim}


    \subsubsection{Plot For Training Data}\label{plot-for-training-data}

\textbf{Red: Ground Truth Values }

\textbf{Blue: Predicted Values}

    \begin{Verbatim}[commandchars=\\\{\}]
{\color{incolor}In [{\color{incolor}74}]:} \PY{c+c1}{\PYZsh{}\PYZsh{} Plotting the Outputs vs Inputs}
         \PY{n}{plt}\PY{o}{.}\PY{n}{plot}\PY{p}{(}\PY{n}{net}\PY{p}{[}\PY{l+m+mi}{0}\PY{p}{]}\PY{p}{,}\PY{n}{net}\PY{p}{[}\PY{l+m+mi}{1}\PY{p}{]}\PY{p}{,}\PY{l+s+s1}{\PYZsq{}}\PY{l+s+s1}{r.}\PY{l+s+s1}{\PYZsq{}}\PY{p}{)}
         
         
         \PY{c+c1}{\PYZsh{}\PYZsh{} Caluclating Predictions from the network}
         \PY{n}{plt}\PY{o}{.}\PY{n}{plot}\PY{p}{(}\PY{n}{net}\PY{p}{[}\PY{l+m+mi}{0}\PY{p}{]}\PY{p}{,} \PY{n}{feed\PYZus{}forward}\PY{p}{(}\PY{n}{net}\PY{p}{[}\PY{l+m+mi}{0}\PY{p}{]}\PY{p}{,}\PY{n}{tr}\PY{p}{[}\PY{l+m+mi}{1}\PY{p}{]}\PY{p}{,}\PY{n}{tr}\PY{p}{[}\PY{l+m+mi}{2}\PY{p}{]}\PY{p}{)}\PY{p}{,}\PY{l+s+s1}{\PYZsq{}}\PY{l+s+s1}{b.}\PY{l+s+s1}{\PYZsq{}}\PY{p}{)}
         
         \PY{n}{plt}\PY{o}{.}\PY{n}{show}\PY{p}{(}\PY{p}{)}
         
         
         \PY{c+c1}{\PYZsh{} Plotting the Error}
         \PY{n}{plt}\PY{o}{.}\PY{n}{plot}\PY{p}{(}\PY{n}{tr}\PY{p}{[}\PY{l+m+mi}{0}\PY{p}{]}\PY{p}{)}
         \PY{n}{plt}\PY{o}{.}\PY{n}{show}\PY{p}{(}\PY{p}{)}
\end{Verbatim}


    \begin{center}
    \adjustimage{max size={0.9\linewidth}{0.9\paperheight}}{output_15_0.png}
    \end{center}
    { \hspace*{\fill} \\}
    
    \begin{center}
    \adjustimage{max size={0.9\linewidth}{0.9\paperheight}}{output_15_1.png}
    \end{center}
    { \hspace*{\fill} \\}
    
    \subsubsection{Generating Test Data}\label{generating-test-data}

    \begin{Verbatim}[commandchars=\\\{\}]
{\color{incolor}In [{\color{incolor}75}]:} \PY{n}{x}\PY{o}{=} \PY{n}{np}\PY{o}{.}\PY{n}{random}\PY{o}{.}\PY{n}{rand}\PY{p}{(}\PY{n}{n\PYZus{}inputs}\PY{p}{,}\PY{n}{dimensions}\PY{p}{)} \PY{c+c1}{\PYZsh{}\PYZsh{} Randomly Genrated Inputs}
         \PY{n}{y}\PY{o}{=} \PY{l+m+mi}{7}\PY{o}{*}\PY{n}{x} \PY{o}{+} \PY{l+m+mi}{3} \PY{o}{+} \PY{n}{np}\PY{o}{.}\PY{n}{random}\PY{o}{.}\PY{n}{randn}\PY{p}{(}\PY{l+m+mi}{1}\PY{p}{)} 
         \PY{n}{y} \PY{o}{=} \PY{n}{np}\PY{o}{.}\PY{n}{sum}\PY{p}{(}\PY{n}{y}\PY{p}{,}\PY{n}{axis}\PY{o}{=}\PY{l+m+mi}{1}\PY{p}{,}\PY{n}{keepdims}\PY{o}{=} \PY{k+kc}{True}\PY{p}{)}
\end{Verbatim}


    \subsubsection{Plot For Test Data}\label{plot-for-test-data}

\textbf{Red: Ground Truth Values }

\textbf{Blue: Predicted Values}

    \begin{Verbatim}[commandchars=\\\{\}]
{\color{incolor}In [{\color{incolor}76}]:} \PY{c+c1}{\PYZsh{}\PYZsh{} Plotting the Outputs vs Inputs}
         \PY{n}{plt}\PY{o}{.}\PY{n}{plot}\PY{p}{(}\PY{n}{net}\PY{p}{[}\PY{l+m+mi}{0}\PY{p}{]}\PY{p}{,}\PY{n}{y}\PY{p}{,}\PY{l+s+s1}{\PYZsq{}}\PY{l+s+s1}{r.}\PY{l+s+s1}{\PYZsq{}}\PY{p}{)}
         
         \PY{c+c1}{\PYZsh{}\PYZsh{} Caluclating Predictions from the network}
         \PY{n}{plt}\PY{o}{.}\PY{n}{plot}\PY{p}{(}\PY{n}{x}\PY{p}{,}\PY{n}{feed\PYZus{}forward}\PY{p}{(}\PY{n}{x}\PY{p}{,}\PY{n}{tr}\PY{p}{[}\PY{l+m+mi}{1}\PY{p}{]}\PY{p}{,}\PY{n}{tr}\PY{p}{[}\PY{l+m+mi}{2}\PY{p}{]}\PY{p}{)}\PY{p}{,}\PY{l+s+s1}{\PYZsq{}}\PY{l+s+s1}{b.}\PY{l+s+s1}{\PYZsq{}}\PY{p}{)}
         
         \PY{n}{plt}\PY{o}{.}\PY{n}{show}\PY{p}{(}\PY{p}{)}
         
         
         \PY{c+c1}{\PYZsh{} Plotting the Error}
         \PY{n}{plt}\PY{o}{.}\PY{n}{plot}\PY{p}{(}\PY{n}{tr}\PY{p}{[}\PY{l+m+mi}{0}\PY{p}{]}\PY{p}{)}
         \PY{n}{plt}\PY{o}{.}\PY{n}{show}\PY{p}{(}\PY{p}{)}
\end{Verbatim}


    \begin{center}
    \adjustimage{max size={0.9\linewidth}{0.9\paperheight}}{output_19_0.png}
    \end{center}
    { \hspace*{\fill} \\}
    
    \begin{center}
    \adjustimage{max size={0.9\linewidth}{0.9\paperheight}}{output_19_1.png}
    \end{center}
    { \hspace*{\fill} \\}
    
    \subsubsection{Important Inferences}\label{important-inferences}

\begin{enumerate}
\def\labelenumi{\arabic{enumi}.}
\tightlist
\item
  The relation of increased inputs to faster convergence also holds true
  here
\item
  As the dimensions increase, more epochs are needed or more training
  samples are needed to get high accuracy.
\item
  Learning rate og 0.001 is again found to e ideal. If we decrease the
  learning rate significantly, the error can aslo shoot up and therefore
  no convergence. With very low learning rate we need higher epochs.
\end{enumerate}

    \subsection{(C) Comments on Hyper
Parameters}\label{c-comments-on-hyper-parameters}

\begin{itemize}
\tightlist
\item
  For the first case of 1D input:

  \begin{itemize}
  \tightlist
  \item
    there were two hyper paramters: \textbf{Learning Rate} and the
    number of \textbf{epochs}
  \item
    Apart from this, \textbf{increasing the number of training inputs}
    also helps reducing the convergence time.
  \item
    Therefore, tuning just three paramters is not too tedious and
    therefore, getting the right paramters is relatively easy.
  \end{itemize}
\item
  For the second case of nD input:

  \begin{itemize}
  \tightlist
  \item
    here there is an additional hyperparamter of \textbf{Dimension}
  \item
    This adds on to the complexity because, by increasing the
    dimensions, more data is required or more epochs are required to
    converge.
  \item
    But still, the number of paramters to tune are 4, which is still
    reasonable and therefore finding the right parmaters is still not
    that difficult
  \end{itemize}
\end{itemize}


    % Add a bibliography block to the postdoc
    
    
    
    \end{document}
